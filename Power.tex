\chapter{Powers and Functions}
\section{Rules} RSG shall create rules and/or sanctions for the governing of students within the Graduate School. Any such rules adopted by RSG shall be presented to the Dean for her review and approval.

\section{Fees}\label{sec:fees} RSG shall levy fees to Students, as approved by majority vote of the 
Student body, pursuant to Article I, Section C of the Constitution. 

\section{Appropriations} RSG shall keep, manage, and appropriate monies collected under Bylaw~\ref{sec:fees} and all other sources of income.

\section{Sponsorship} RSG shall appropriate funds for programs designed to enhance and improve the Graduate Student 
community pursuant to the rules contained in these bylaws. 

\section{Lobbying} RSG shall lobby for the interests of Students within the University and externally to federal, state, and local entities as deemed appropriate by the 
Board. 

\section{Representation} RSG shall represent the interests of the Student Body to all 
school, university, and external entities. 

\section{Appointments} RSG shall be the sole appointer of Student representatives for all committees requesting graduate student views and input. 

\section{Elections} Each year, RSG shall hold two elections for candidates to serve on the RSG Board. The Board may, by a simple majority vote order additional elections for other purposes. Elections may include referenda. 

\section{Bylaws} RSG shall maintain these Bylaws in order to exercise their powers and to carry 
out the functions described herein. 

\section{Summer Operations} RSG shall remain in force throughout the entire calendar 
year.

\section{Referenda} RSG shall send to the student body such questions as it deems 
necessary by a majority vote. Such questions can be informational, such as to 
gauge the general opinion of the student body on a given issue, or binding such that 
the outcome of the vote will be binding on RSG policy decisions.
