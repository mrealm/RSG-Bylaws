\chapter{Infrastructure}
\section{Committee Descriptions}
\begin{enumsubsection}
\item{Budgetary Committee (BC)}\\
\begin{tabular}{ll}
Composition & 7 or more members\\
Required Members:& Treasurer (Chair)\\
 & President\\
 & Vice President\\
 & 4 current Board members, 1 from each division \\
 & Such additional board members as the Vice-president\\
 & may slate. The total size of the committee may not\\
 & exceed ten members.
\end{tabular}
\begin{enumsubsubsection}
\itemnotoc Quorum: A minimum of five members must be present (in person or electronically) for the committee to conduct business. In the event that there are ten members on the committee, quorum shall instead be set at six.
\itemnotoc Representatives on the Budgetary Committee: In addition to the three executives, the Budgetary Committee is required to have at least one member from each of the four Rackham divisions.  No graduate program within a division can be represented more than once, excluding the graduate student body executives, without 2/3rds approval of the board.  In the event that there is no sitting representative from a division, or the member or members from that division cannot serve, the Vice-President may nominate a board member from a graduate program that is already represented on the Committee.
\itemnotoc   In the event the Budgetary Committee is unable to attain a quorum for a meeting, and the business before the committee is pressing, the Treasurer will notify the President and the Board shall assume the Budgetary Committee's duties and powers detailed in \ref{sec:bud_resp} of these Bylaws to resolve the committee's pending time-sensitive business.


\itemnotoc\label{sec:bud_resp} The Budgetary Committee is responsible for receiving, reviewing, and 
authorizing all funding requests from Student organizations by majority 
vote. Student Groups requesting funding may request to present their 
proposal to the Committee.
\itemnotoc The Budgetary Committee, with the action power described in \ref{sec:bud_resp} of the Bylaws, is required to maintain minutes. These minutes shall 
include the individual voting records of all members for all funding 
requests. Minutes shall be taken by the treasurer and kept on the RSG 
website.. Minutes shall be presented to the Board consistent with the 
general rules for committee minutes as provided for in these bylaws.
\itemnotoc The Budgetary Committee may hold its fiscal deliberations 
electronically. In such a case, all committee members will be required to 
participate in the discussion in a timely manner. Individuals that do not 
contribute to a discussion, without the approval of the Treasurer shall 
automatically relinquish their vote on the pending matter(s). Repeated 
lack of participation shall be cause for removal from the Committee by 
the Board. Unless otherwise specified by the Treasurer, a timely manner 
shall be considered to be 5 business days. 
\itemnotoc Student organizations granted funding by the Board shall be required 
to produce itemized receipts and an Event Report to the Treasurer 
\emph{prior} to receiving funds. 
\itemnotoc All rules contained in these bylaws as well as in the allocation email 
from the RSG Treasurer must be adhered to in order to receive 
reimbursements. Article VIII specifically addresses organization 
funding requests.
\itemnotoc Five percent (5\%) of the estimated yearly budget shall at all times be 
kept in reserve for emergency costs. These funds shall not be 
considered available for funding Student organizations. Use of this 
reserve shall only be authorized by a majority vote of the Board and 
the concurrence of both the Treasurer and the President. 
\itemnotoc Within 3 business days of a funding decision, the Treasurer or his/her 
designee shall notify the requesting organization of the committee's 
decision, including its rationale and any stipulations provided by the 
committee. The requesting organization shall also be notified of its 
option to appeal the committee's funding decision to the Board. 
Appeals must be submitted electronically to the President no later than 
5 business days after the committee's original decision is emailed and 
no less than 7 business days before the requesting organization's 
proposed event is scheduled to commence. 
\itemnotoc Funding limit. In the instance where the Budgetary Committee votes to 
allocate over \$1,000 to an organization or event, the funding request 
and the committee's vote will be presented to the Board at the next 
regularly scheduled Board meeting. The Board will be given an 
opportunity to ask questions regarding the event and the funding 
process. Members of the Board may, by a simple majority vote deny 
the budgetary committee's allocation. In the event that an allocation is 
denied by the Board, the committee will re-convene to reconsider its 
allocation in light of the Board's vote and its directives to the 
committee. 
\end{enumsubsubsection}
\item{Student Life Committee (SLC)}\\
\begin{tabular}{ll}
Composition & Open Enrollment\\
Required Members:& 4 current Board members
\end{tabular}
\begin{enumsubsubsection}
\itemnotoc The Student Life Committee shall address all non-academic needs and 
concerns of the Student Body. Additionally, the committee shall serve to 
help unite the Student body through the hosting of large communal events 
designed to stimulate student social interaction. Specific attention shall be 
given to creating/hosting events that are inter-departmental, that strive to 
create a welcoming atmosphere for minority groups and that help to unite 
both central and northern sub-campus locations. 

\itemnotoc The Student Life Committee shall be required to host at least three major Student social events per winter and fall semesters. One major social event 
during the spring and summer months shall be held, funding permitting. 
\itemnotoc The Community Outreach and Social Action Subcommittee. SLC shall maintain a standing sub-committee that shall work collaboratively with the full student life committee, other RSG committees, as well as non-RSG related service organizations in order to provide a diverse array of volunteer opportunities to the graduate student body, with focus on providing service to the greater Ann Arbor citizenry. 
\end{enumsubsubsection}
\item{Academic Affairs Committee (AAC)}\\
\begin{tabular}{ll}
Composition & Open Enrollment\\
Required Members:& 3 current Board members
\end{tabular}
\begin{enumsubsubsection}
\itemnotoc The Academic Affairs Committee shall address academic \& 
programmatic needs of the Student body to the Rackham 
Administration. These issues shall include, but are not limited to, 
financial aid, fellowships, grants, awards, summer funding, and 
academic policies. 

\end{enumsubsubsection}
\item{Elections Committee (EC)}\\
\begin{tabular}{lp{3.5in}}
Composition & 4 or more members\\
Required Members:& Vice President or Treasurer (Chair)* (The Vice 
President shall chair the committee in the fall term. The Treasurer shall 
chair the committee in the Winter term.) \\
 & 4 current Board members (1 from each Division)\\
 & 1 Election Director
\end{tabular}
\begin{enumsubsubsection}
\itemnotoc The Elections Committee shall provide active advisement and 
procedural help to the Election Director, aiding him/her in running a 
smooth and timely election each fall and winter term. 
\itemnotoc The Elections Committee shall receive and be copied on any and all 
elections related correspondence to ensure transparency between the 
Elections Director and the committee.
\itemnotoc The Elections Committee shall propose any changes to RSG's election 
policy no later than 4 weeks prior to a given election. 
\itemnotoc The Elections Committee shall assist the executives in the solicitation for 
and the selection of an election director. 

\itemnotoc In the event that any member of the Elections Committee decides to run for election, a replacement shall be appointed by the President, with the approval of the Board.
\end{enumsubsubsection}


\item{Legislative Affairs Committee (LAC)}\\
\begin{tabular}{ll}
Composition & Open Enrollment\\
Required Members:& President (must be chair or co-chair)\\
 & Vice President\\
 & 4 current Board members
\end{tabular}
\begin{enumsubsubsection}
\itemnotoc The Legislative Affairs Committee will collaborate with external 
advocacy organizations such as the Student Advocates for Graduate 
Education (SAGE) coalition, the Michigan Association of Graduate and 
Professional Students (MAGPS), and the National Association of
Graduate and Professional Students (NAGPS) to promote the legislative 
agenda of the student body. 
\itemnotoc The Legislative Affairs Committee will work with the City of Ann Arbor 
and Washtenaw County to promote the local interests of the Student 
Body. 
\itemnotoc The Legislative Affairs Committee will work with the State and Federal 
Governments, as well as other entities beyond the University to promote 
the interests of the Student Body. 
\end{enumsubsubsection}
\end{enumsubsection}

\section{Committee Composition and Powers}
\begin{enumsubsection}
\itemnotoc The Vice President will, with the advice and consent of the President and 
Treasurer, nominate members of the Board to sit on RSG's various 
committees by the 2\textsuperscript{nd} meeting of each semester. 
\itemnotoc Each committee shall have a single chair or co-chairs. 
\itemnotoc Chairs are voted on in committee with the exception of the Budgetary 
Committee and the Legislative Committee and will be presented to the 
Board for Confirmation by its (the Board's) 3rd meeting each semester. 
\itemnotoc Committees with open enrollment may include any number of non-RSG 
members or Associate Members at the discretion of the committee chair, 
the vice president, the president, or with the approval of the Board. 
\itemnotoc No committee shall act on issues outside of its stipulated responsibilities 
described herein without prior authorization from the Board, or as directed 
by the President or Vice-President. 
\itemnotoc Committees may be created on a temporary basis for special projects by 
either the President through an executive order, or by a majority resolution 
as approved by the Board. A resolution or executive order that creates an 
ad-hoc committee must specify the committee's charge, composition, 
lifespan, chair, operating procedures, and privileges (financial, etc). 
\end{enumsubsection}
\section{Committee Responsibilities}
\begin{enumsubsection}
\item{Meetings}
\begin{enumsubsubsection}
\itemnotoc Committees shall meet at least bi-weekly and at the discretion of 
the chair. 
\itemnotoc The chair of each committee shall establish the time and 
location of each meeting, providing at least 48 hours' notice to 
committee members, and provide an electronic copy of the 
committee's preliminary agenda to committee members and 
upon a written request, to members of the student body. 
\end{enumsubsubsection}
\item{Minutes}
\begin{enumsubsubsection}
\itemnotoc Each committee chair shall ensure that minutes are 
recorded for each meeting. These minutes will be sent to 
the committee's members within 5 days of the meeting 
for review and approval. 
\itemnotoc Each committee chair will submit approved minutes to 
the Board within 2 days of approval by the committee.
\itemnotoc Minutes will be presented in a format provided by the 
Vice President.
\itemnotoc Minutes will include meeting attendance as well as an 
accurate record of the committee's deliberations, 
decisions, and future plans.
\end{enumsubsubsection}
\item{Oral Reports}
\begin{enumsubsubsection}
\itemnotoc Each committee chair shall be required to report any and all 
committee activities to the Board at each general meeting. 
\itemnotoc The chair of each committee shall be responsible for maintaining 
a written record of meeting attendance. Attendance is required 
and must be emailed to the Vice President within one week the 
meeting 
\itemnotoc The chair(s) of each committee may appoint a vice-chair from 
within the committee's membership. The selection / election of a 
vice chair will be reported to the Board. A vice chair shall be 
responsible for taking minutes and for chairing in the elected 
chair's absence as well as for any other tasks so delegated by the 
chair. 
\end{enumsubsubsection}
\item{Budget}
\begin{enumsubsubsection}
\itemnotoc Each committee must present to the Treasurer a budget no later 
than the second regular meeting of the term. Committee chair(s) 
will take the lead in forming these budgets under the guidance of 
the Vice President.
\end{enumsubsubsection}
\end{enumsubsection}
\section{Recall of Committee Members}
\begin{enumsubsection}
\itemnotoc Any committee member including the chair may be removed from 
their position on the committee by a majority vote of the Board in the 
event of: 
\begin{enumsubsubsection}
\itemnotoc The accumulation of 2 unexcused absences in a given academic 
term. Absences may be excused by the chair when requested in 
writing. 
\itemnotoc Failure to report committee activities to the Board for two 
consecutive meetings (where committee activities have occurred). 
\itemnotoc A motion by another member or chair of the committee, and a 
majority vote in favor by the Board. 
\end{enumsubsubsection}
\itemnotoc Any committee member including the chair who is up for recall by 
majority vote shall be given an opportunity to address the Board. 
Following this speaking opportunity, the Board will vote by secret ballot. 
The ballots will be counted by the executive officers. 
\itemnotoc The President may, with the consent of either the Vice President or
the Treasurer temporarily remove a committee member or chair. 
Temporary removal shall be voided upon the next regularly scheduled 
meeting of the Board. 
\end{enumsubsection}
